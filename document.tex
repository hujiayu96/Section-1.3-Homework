\documentclass[12pt,a4paper]{article}
\usepackage[latin1]{inputenc}
\usepackage{amsmath}
\usepackage{amsfonts}
\usepackage{amssymb}
\usepackage{graphicx}
\usepackage{float}
\usepackage[left=2.4cm, right=2.4cm, top=2.00cm, bottom=2.00cm]{geometry}
\usepackage{listings}
\usepackage[dvipsnames]{xcolor}
\usepackage{sectsty}
\usepackage{caption}
\usepackage{indentfirst}
\renewcommand{\labelenumi}{\textbf{(\alph{enumi})}}
\title{Section 1.3 Homework}
\author{Jiayu Hu}

\boldmath

\setlength{\parskip}{0.3\baselineskip}

\sectionfont{\LARGE\color{Blue}}
\subsectionfont{\Large\color{Blue}}
\subsubsectionfont{\large\color{Blue}}

\linespread{1.3}
\setlength\parindent{0em}
\begin{document}
\bfseries
\maketitle

\section*{Example 5}
Let $f(n)=1+3+5+...+(2n-1)$\\
Let $p(n)$ be the statement $f(n)=n^2$\\
When $n=1$, we can get $1=1^2$. So, $p(n)$ is true at $n=1$\\
We now assume $\forall k, p(k)$\\
Our goal is to show $\forall k, p(k)\Rightarrow p(k+1)$\\
For $n=k$, we have $f(k)=k^2$\\
So $f(k+1)=f(k)+[2*(k+1)-1]=k^2+2k+1=(k+1)^2$\\
We have shown that $\forall k, p(k)\Rightarrow p(k+1)$.\\  Therefore, we have shown that $\forall n, p(n)\ \blacksquare$

\section*{Example 7}
If $x$ and $y$ are positive odd integers, then we can express $x$ and $y$ as $x=2m+1, y=2n+1$ where $m$ and $n$ are nonnegative integers. Then, we obtain
\begin{eqnarray*}
\begin{aligned}
xy&=(2m+1)(2n+1)&\text{(by definition of $x$ and $y$)}\\
  &=4mn+2m+2n+1 &\text{(expand the expression)}\\
  &=2(2mn+m+n)+1&\text{($2$ is a common factor of $2mn$,\ $m$ and $n$)} \\
  &=2k+1 &\text{($k=2mn+m+n$ is also a positive integer)}\\
  &=Old\ Positive\ Integer &\text{(by definition of a positive odd integer)}\\
  &\blacksquare
\end{aligned}
\end{eqnarray*}

\section*{Example 9}

\begin{tabular}{|c|c|c|c|}
\hline
$p$ & $q$ & $q\Rightarrow p$ & $p\Rightarrow q$\\ \hline
$T$ & $T$ & $T$ & $T$\\ \hline
$T$ & $F$ & $T$ & $F$\\ \hline
$F$ & $T$ & $F$ & $T$\\ \hline
$F$ & $F$ & $T$ & $T$\\ \hline
\end{tabular}
\\\\
Conclusion: $q\Rightarrow p$ is not logically equivalent to $p\Rightarrow q$

\section*{Example 10}
\begin{tabular}{|c|c|c|c|c|c|}
	\hline
	$p$ & $q$ & $\sim p$ & $\sim q$ & $\sim p\Rightarrow \sim q$ & $p\Rightarrow q$\\ \hline
	$T$ & $T$ & $F$ & $F$ & $T$ & $T$\\ \hline
	$T$ & $F$ & $F$ & $T$ & $T$ & $F$\\ \hline
	$F$ & $T$ & $T$ & $F$ & $F$ & $T$\\ \hline
	$F$ & $F$ & $T$ & $T$ & $T$ & $T$\\ \hline
\end{tabular}
\\\\
Conclusion: $\sim p\Rightarrow \sim q$ is not logically equivalent to $p\Rightarrow q$

\section*{Example 11}
\begin{enumerate}
\item If it is snowing, then the temperature outside is less than one-hundred degrees Fahrenheit.

Contrapositive: If the temperature outside is not less than one-hundred degrees Fahrenheit, then it not snowing. $(T)$

Converse: If the temperature outside is less than one-hundred degrees Fahrenheit, then it is snowing. $(F)$

Inverse: If it is not snowing, then the temperature outside is not less than one-hundred degrees Fahrenheit. $(F)$


\item If an animal has feet, then it can walk.

Contrapositive: If an animal can not walk, then it does not have feet. $(F)$

Converse: If an animal can walk, then it has feet. $(T)$

Inverse: If an animal does not have feet, then it can not walk. $(T)$
\end{enumerate}
\end{document}